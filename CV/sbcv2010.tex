%%%%%%%%%%%%%%%%%%%%%%%%%%%%%%%%%%%%%%%%%%%%%%%%%%%%%%%%%%%%%%%%%%%%%%%%
%%%%%%%%%%%%%%%%%%%%%% Simple LaTeX CV Template %%%%%%%%%%%%%%%%%%%%%%%%
%%%%%%%%%%%%%%%%%%%%%%%%%%%%%%%%%%%%%%%%%%%%%%%%%%%%%%%%%%%%%%%%%%%%%%%%

%%%%%%%%%%%%%%%%%%%%%%%%%%%%%%%%%%%%%%%%%%%%%%%%%%%%%%%%%%%%%%%%%%%%%%%%
%% NOTE: If you find that it says                                     %%
%%                                                                    %%
%%                           1 of ??                                  %%
%%                                                                    %%
%% at the bottom of your first page, this means that the AUX file     %%
%% was not available when you ran LaTeX on this source. Simply RERUN  %%
%% LaTeX to get the ``??'' replaced with the number of the last page  %%
%% of the document. The AUX file will be generated on the first run   %%
%% of LaTeX and used on the second run to fill in all of the          %%
%% references.                                                        %%
%%%%%%%%%%%%%%%%%%%%%%%%%%%%%%%%%%%%%%%%%%%%%%%%%%%%%%%%%%%%%%%%%%%%%%%%

%%%%%%%%%%%%%%%%%%%%%%%%%%%% Document Setup %%%%%%%%%%%%%%%%%%%%%%%%%%%%

% Don't like 10pt? Try 11pt or 12pt
\documentclass[10pt]{article}

% This is a helpful package that puts math inside length specifications
\usepackage{calc}

% Layout: Puts the section titles on left side of page
\reversemarginpar

%
%         PAPER SIZE, PAGE NUMBER, AND DOCUMENT LAYOUT NOTES:
%
% The next \usepackage line changes the layout for CV style section
% headings as marginal notes. It also sets up the paper size as either
% letter or A4. By default, letter was used. If A4 paper is desired,
% comment out the letterpaper lines and uncomment the a4paper lines.
%
% As you can see, the margin widths and section title widths can be
% easily adjusted.
%
% ALSO: Notice that the includefoot option can be commented OUT in order
% to put the PAGE NUMBER *IN* the bottom margin. This will make the
% effective text area larger.
%
% IF YOU WISH TO REMOVE THE ``of LASTPAGE'' next to each page number,
% see the note about the +LP and -LP lines below. Comment out the +LP
% and uncomment the -LP.
%
% IF YOU WISH TO REMOVE PAGE NUMBERS, be sure that the includefoot line
% is uncommented and ALSO uncomment the \pagestyle{empty} a few lines
% below.
%

%% Use these lines for letter-sized paper
%\usepackage[paper=letterpaper,
%            %includefoot, % Uncomment to put page number above margin
%            marginparwidth=1.2in,     % Length of section titles
%            marginparsep=.05in,       % Space between titles and text
%            margin=1in,               % 1 inch margins
%            includemp]{geometry}

% Use these lines for A4-sized paper
\usepackage[paper=a4paper,
            %includefoot, % Uncomment to put page number above margin
            marginparwidth=30.5mm,    % Length of section titles
            marginparsep=1.5mm,       % Space between titles and text
            margin=16mm,              % 25mm margins
            includemp]{geometry}

%% More layout: Get rid of indenting throughout entire document
\setlength{\parindent}{0in}

%% This gives us fun enumeration environments. compactenum will be nice.
\usepackage{paralist}

%% Reference the last page in the page number
%
% NOTE: comment the +LP line and uncomment the -LP line to have page
%       numbers without the ``of ##'' last page reference)
%
% NOTE: uncomment the \pagestyle{empty} line to get rid of all page
%       numbers (make sure includefoot is commented out above)
%
\usepackage{fancyhdr,lastpage,soul}
\pagestyle{fancy}
%\pagestyle{empty}      % Uncomment this to get rid of page numbers
\fancyhf{}\renewcommand{\headrulewidth}{0pt}
\fancyfootoffset{\marginparsep+\marginparwidth}
\newlength{\footpageshift}
\setlength{\footpageshift}
          {0.5\textwidth+0.5\marginparsep+0.5\marginparwidth-2in}
\lfoot{\hspace{\footpageshift}%
       \parbox{4in}{\, \hfill %
                    \arabic{page} of \protect\pageref*{LastPage} % +LP
%                    \arabic{page}                               % -LP
                    \hfill \,}}

% Finally, give us PDF bookmarks
\usepackage{color,hyperref}
\definecolor{darkblue}{rgb}{0.0,0.0,0.3}
\hypersetup{colorlinks,breaklinks,
            linkcolor=darkblue,urlcolor=darkblue,
            anchorcolor=darkblue,citecolor=darkblue}

%%%%%%%%%%%%%%%%%%%%%%%% End Document Setup %%%%%%%%%%%%%%%%%%%%%%%%%%%%


%%%%%%%%%%%%%%%%%%%%%%%%%%% Helper Commands %%%%%%%%%%%%%%%%%%%%%%%%%%%%

% The title (name) with a horizontal rule under it
%
% Usage: \makeheading{name}
%
% Place at top of document. It should be the first thing.
\newcommand{\makeheading}[1]%
        {\hspace*{-\marginparsep minus \marginparwidth}%
         \begin{minipage}[t]{\textwidth+\marginparwidth+\marginparsep}%
                {\large \bfseries #1}\\[-0.15\baselineskip]%
                 \rule{\columnwidth}{1pt}%
         \end{minipage}}

% The section headings
%
% Usage: \section{section name}
%
% Follow this section IMMEDIATELY with the first line of the section
% text. Do not put whitespace in between. That is, do this:
%
%       \section{My Information}
%       Here is my information.
%
% and NOT this:
%
%       \section{My Information}
%
%       Here is my information.
%
% Otherwise the top of the section header will not line up with the top
% of the section. Of course, using a single comment character (%) on
% empty lines allows for the function of the first example with the
% readability of the second example.
\renewcommand{\section}[2]%
        {\pagebreak[2]\vspace{1.3\baselineskip}%
         \phantomsection\addcontentsline{toc}{section}{#1}%
         \hspace{0in}%
         \marginpar{
         \raggedright \scshape #1}#2}

% An itemize-style list with lots of space between items
\newenvironment{outerlist}[1][\enskip\textbullet]%
        {\begin{enumerate}[#1]}{\end{enumerate}%
         \vspace{-.6\baselineskip}}

% An environment IDENTICAL to outerlist that has better pre-list spacing
% when used as the first thing in a \section
\newenvironment{lonelist}[1][\enskip\textbullet]%
        {\vspace{-\baselineskip}\begin{list}{#1}{%
        \setlength{\partopsep}{0pt}%
        \setlength{\topsep}{0pt}}}
        {\end{list}\vspace{-.6\baselineskip}}

% An itemize-style list with little space between items
\newenvironment{innerlist}[1][\enskip\textbullet]%
        {\begin{compactenum}[#1]}{\end{compactenum}}

% To add some paragraph space between lines.
% This also tells LaTeX to preferably break a page on one of these gaps
% if there is a needed pagebreak nearby.
\newcommand{\blankline}{\quad\pagebreak[2]}

%%%%%%%%%%%%%%%%%%%%%%%% End Helper Commands %%%%%%%%%%%%%%%%%%%%%%%%%%%

%%%%%%%%%%%%%%%%%%%%%%%%% Begin CV Document %%%%%%%%%%%%%%%%%%%%%%%%%%%%

\begin{document}
\pagestyle{empty}
\makeheading{Sudchai Boonto}

% ==================================================================================== %
\section{Contact Information}
%
% NOTE: Mind where the & separators and \\ breaks are in the following
%       table.
%
% ALSO: \rcollength is the width of the right column of the table
%       (adjust it to your liking; default is 1.85in).
%
\newlength{\rcollength}\setlength{\rcollength}{2.6in}%
%,
\begin{tabular}[t]{@{}p{\textwidth-\rcollength}p{\rcollength}}
\multicolumn{1}{l}{\hskip-0.2cm\href{http://www.inc.eng.kmutt.ac.th}
     {Department of Control System and Instrumentation}}       & \textit{Tel:} +662 470 9094 \\
\multicolumn{1}{l}{\hskip-0.2cm\href{http://www.inc.eng.kmutt.ac.th}
     {Engineering}}                                            &  \\
\multicolumn{1}{l}{\hskip-0.2cm\href{http://www.kmutt.ac.th/}{King Mongkut's University of Technology Thonburi}}                          & \textit{Fax:} +662 470 9100 \\
126 Prachautid Road,                       & \textit{E-mail:}
\href{mailto:boonto@tu-harburg.de}{\mbox{sudchai.boo@kmutt.ac.th}}\\
 Bangmod, Tungkru, 10140 & \textit{WWW:}
\href{http://www.kmutt.ac.th/}{www.kmutt.ac.th}\\


\end{tabular}
% ================================================================= %

\section{Research Interests}
%
Robust Control, Linear Parameter-Varying Control, Mechatronic Systems, Feedforward Robust Control, Robust Repetitive Control, System Identification, Convex Optimization Applications

% ============================================================== %
\section{Education}
%
\href{http://www.tuhh.de}{\textbf{Hamburg University of Technology (TUHH)}}
\begin{outerlist}
\item[] Dr.-Ing. in Automatic Control Engineering 2011\\
\end{outerlist}

%
\href{http://www.man.ac.uk/}{\textbf{The University of Manchester Institute
                   of Science and Technology (UMIST)  }},
Now is The University of Manchester, UK
\begin{outerlist}
\item[] M.Sc.,
        \href{http://www.eee.manchester.ac.uk/research/groups/cs/}
             {Advanced Control}, March 2000\\
\end{outerlist}


\href{http://www.kmutt.ac.th/}{\textbf{King Mongkut's University of Technology Thonburi}}, Bangkok, Thailand
\begin{outerlist}
  \item[] B.Eng.,
        \href{http://www.kmutt.ac.th/organization/Engineering/Electrical/}{Electrical Engineering}, May 1995
\end{outerlist}
%\item[] B.S.,
%        \href{http://www.ece.osu.edu/}
%             {Electrical and Computer Engineering}, June 2004
        %\begin{innerlist}
%        \item \emph{Magna cum Laude}, With Honors in Engineering
%        \item Electrical specialization (emphasis on electromagnetics and digital computers)
%        \item Minor in \href{http://www.cse.ohio-state.edu/}
%                            {Computer and Information Systems}
%              (programming and algorithms track)
%        \end{innerlist}

% ==================================================================================== %

\section{Awards}
%
The Royal Thai Government (Ministry of University Affairs)
full scholarship to pursue study in Germany (Doctoral degree)

% ==================================================================================== %
\section{Academic Experience}
\href{http://www.kmutt.ac.th/}{\textbf{King Mongkut's University of Technology Thonburi (KMUTT)}}, Bangkok Thailand\\[0.5cm]
\emph{Assistant Professor} \hfill \textbf{2013--present}\\[0.5cm]
\emph{Lecturer} \hfill  \textbf{1995--2013}\\[0.5cm]
\emph{Undergraduate }
\begin{outerlist}
    \item INC 102 Instrumentation and Process Control
	\item INC 111 ฺBasic Engineering Circuit Analysis
	\item INC 151 Circuit Analysis by Engineering's Software Practice
    \item INC 211 Mathematics for Signals and Systems
    \item INC 221 Electronics Devices and Circuit Design
    \item INC 231 Electrical Measurement
    \item INC 251 Digital and Electronics Laboratory I
    \item INC 341 Feedback Control System
    \item INC 354 Process Instrumentation Laboratory
    \item INC 481 System Dynamics and Modelling\\
\end{outerlist}
\emph{Graduate}
\begin{outerlist}
    \item INC 521 System Identification
    \item INC 691 Linear and Nonlinear System Identification
    \item INC 692 Robust Control
    \item INC 693 System Dynamics and Modelling
    \item INC 694 Neural Network and Its Applications
    \item EEE 600 System Analysis Techniques\\
\end{outerlist}
\href{http://www.tuhh.de/rts/}{\textbf{Hamburg University of Technology (TUHH)}}, Hamburg Germany\\
\emph{Teaching Assistant} \hfill \textbf{2003--2009}
\begin{outerlist}
  \item  Neural and Genetic Computing for Control of Dynamic Systems
\end{outerlist}
% ==================================================================================== %
\section{Research Activities}
\begin{outerlist}
  \item ``Robust PID Control for Grid-Control Three Phase Inverter using Convex-Concave Optimization'', {Government Budget Grant} 2017-2018 {[Principal Investigator]}
  \item ``Transient response improvement of  PVC process with disturbance from reflux condenser'', {SCG Chemical} 2016-2017 {[Principal Investigator]}
  \item``Two-Degree-of-Freedom $\mathcal{H}_\infty$ Repetitive Control for Grid-Connected Inverter'', {Research Strengthening Project of the Faculty of Engineering KMUTT}  2014-2016  {[Principal Investigator]}
  \item ``Advanced Control System Design for Power Electronics and Motor Drives using a New Heuristic Optimization Algorithm,'' Research Strengthening Project of the Faculty of Engineering, King Mongkut’s University of Technology Thonburi, 2014-2017 {[Team Member]}

\end{outerlist}
\textbf{Before 2011}
\begin{outerlist}
  \item ``Prototype development of the noise and vibration spectrum analyzer for industrial process'', KMUTT, 2002, {[Team Member]}
  \item ``Development of the seed moisture meter prototype using a microwave frequency range non-destructive method'', KMUTT, 2003, {[Team Member]}
\end{outerlist}
% ==================================================================================== %
\section{Scholarships for Students}
\begin{outerlist}
  \item \emph{M.Eng. Scholarship} ``$\mathcal{H}_\infty$ Power Control for Grid--Connected Inverter'' Energy Policy and Planning Office (EPPO) years 2012-2014
  \item \emph{M.Eng. Scholarship} ``2DoF Robust Repetitive Control for Grid--Connected Inverter'' Energy Policy and Planning Office (EPPO) years 2015-2017
\end{outerlist}
% ==================================================================================== %
\section{Professional Service}
\textbf{IEEE Control System Chapter Thailand}
\begin{outerlist}
    \item Secretary 2016-present
\end{outerlist}
\vspace{1em}
\textbf{Conference Organizing Committee}
\begin{outerlist}
    \item EECON-25
\end{outerlist}
\vspace{1em}
\textbf{Technical Program Chair}
\begin{outerlist}
    \item ICITEE 2017
\end{outerlist}
\vspace{1em}
\textbf{Reviewer of}
\begin{outerlist}
    \item IEEE 2007 Conference on Automation Science and Engineering, TENCON 2014, iEECON 2014
    \item MCS 2009, CDC 2009-2013,2015-2016, IFAC world congress 2011, ECTI-CON 2011,2013-2014,2016, EECON 29, 35-36, ACC 2010-2013,2015, ECC 2015, ICIEA 2017, ASCC 2017, ICITEE 2017
    \item IFAC 8th Safeprocess 2012, SEATUC 2012, TSME-ICoME 2013
    \item KMUTT R\&D Journal, Engineering Journal (EJ), Thammasat International Journal of Science and Technology (TIJSAT)
    \item International Journal of Robust and Nonlinear Control, International Journal of Adaptive Control and Signal Processing
\end{outerlist}
% ==================================================================================== %
\section{Journal Publication}
\textbf{International Journal}
\begin{enumerate}
  \item  {R. K\"{o}rlin, \textbf{S. Boonto}, H. Werner, U. Starossek, `` LMI-based Gain Scheduling for Bridge Flutter Control using Eccentric Rotational Actuators,'' \emph{Optimal Control, Applications and Methods}, Vol. 33, No. 4, 2012, pp. 488--500. (\textbf{Impact Factor = 1.06 in 2012)} }
  \item   T. Nuchkrua, W. Kornmaneesang, S.-L. Chen, and \textbf{S. Boonto},``Precision Contouring Control of 5 DOF Dual-arm Robot Manipulators with Holonomic Constraints'', \emph{IEEE transaction on control system technology} (submitted)
  \item T. Nuchkrua, S.-L. Chen, \textbf{S. Boonto}, ``Adaptive Contouring Control for High-precision 5 DoF Robot Manipulators under Various Environments,'' \emph{Asian Journal of Control }(submitted)
  \item T. Nuchkrau, W. Kornmaneesang, S.-L. Chen, and \textbf{S. Boonto}, ``High-precision Control of Dual-arm Robot Manipulators with Holonomic Constraints,'' \emph{IEEE Robotics and Automation Letters}  (submitted)
  \item \textbf{S. Boonto}, and T. Nuchkrua, ``Non-linear Control for Thermal System of Metal Hydride,'' \emph{International Journal of Hydrogen Energy} (accepted)
\end{enumerate}
\textbf{National Journal}
\begin{enumerate}
  \item {V. Sittiarttakorn and \textbf{S. Boonto}, ``Hybrid Engine Model Using a Stirling Engine and a DC Motor,''\emph{Journal of the Japan Society of Applied Electromagnetic and Mechanics}, Vol. 23, No. 3, 2015, pp. 563--566.}
\end{enumerate}
% ============================================================= %
\section{Conference Presentations}
\textbf{International Conference}
\begin{enumerate}
    \item T. Nuchkrau, W. Kornmaneesang, S.-L. Chen, and \textbf{S. Boonto}, ``Precision Contouring Control of 5 DOF Dual-arm Robot Manipulators with Holonomic Constraints,'' In Proceedings of 2017 Asian Control Conference (ASCC 2017), Gold Coast Convention Centre, Australia, December 17--20, 2017 (to appear)
    \item V. Sittiarttakorn, and \textbf{S. Boonto}, ``Mobile Robot Multi-Paths Tracking Control Using Optical Coding'', In Proceedings of the 4th International Conference on Applied Electrical and Mechanical Engineering 2017 (ICAEME 2017), Nongkhai Thailand, August 31–September 2, 2017 (Best paper award)
    \item T. Nuchkrua, W. Kornmaneesang, S.-L. Chen, \textbf{S. Boonto}, ``A Novel Coordinate Framework of 5-DoF Dual-arm Robot Manipulators for High Precision Path-contouring Control'', In Proceedings of The 13th Conference on Automation Science and Engineering (CASE 2017), Xi'an, China, August 20–23, 2017, (to appear)
    \item T. Nuchkrua, S.-L. Chen, \textbf{S. Boonto},``A Novel Technique of Dual-arm Robot Manipulators: Path-contouring Control Problem,'' In \emph{Proceeding of the 13thIEEE International Conference on Control \& Automation (ICCA 2017)}, Ohrid, Macedonia, July 3–6, 2017, pp. 867--871
    \item P. Phowanna, \textbf{S. Boonto}, E. Mujjalinvimut, M. Konghirun, W. Lenwari, ``Improved Performance of Sliding Mode Observer Using Parameter Adaptation in Sensorless IPMSM Drive,'' In \emph{Proceedings of The 12th IEEE Conference on Industrial Electronics and Applications (ICIEA 2017)}, Siem Reap, Cambodia, June 18-20, 2017
    \item T. Nuchkrua, S.-L. Chen, \textbf{S. Boonto}, ``Adaptive Contouring Control for High-precision 5 DoF Robot Manipulators under Various Environments,'' In \emph{Proceedings of the 2016 International Automatic Control Conference (CACS 2016)}, Evergreen Laurel Hotel,Taichung, Taiwan, Novemver 9-11, 2016
    \item W. Sintanavevong, \textbf{S. Boonto}, S. Naetiladdanon, ``Robust Repetitive Control with Feedforward Scheme for Stand-Alone Inverter,'' In \emph{Proceedings of  the 16th International Conference on Control, Automation and Systems}, HICO, Gyeongju, Korea, October 16-19, 2016, pp. 359--364
    \item C. Thabthimratthana, S. Saelim, S. Tiewcharoen, \textbf{S. Boonto}, ``Robust PID Controller Design Using Convex-Concave Optimization: Application to an Unstable System,'' To appear in \emph{Proceedings of  the 16th International Conference on Control, Automation and Systems}, HICO, Gyeongju, Korea, October 16-19, 2016, pp. 638--643
    \item P. Phowanna, \textbf{S. Boonto},  M. Konghirun, ``Online Parameter Identification Method for IPMSM Drive with MTPA,'' In \emph{Proceedings of  the 18th International Conference on Electrical Machines and Systems}, Pattaya City, Thailand, October 25-28, 2015
    \item V. Sittiarttakorn, \textbf{S. Boonto}, ``Hybrid Modeling Using a Stirling Engine and a {DC} Motor,'' In \emph{Proceedings of the Asia-Pacific Symposium on Applied Electromagnetic and Mechanics}, National Chung Hsing University, Taichung, Taiwan, July 23-25, 2014
    \item W. Sriart, \textbf{S. Boonto}, S. Naetiladdanon, W. Lenwari, ``Grid Connected Inverter Control by Two-Degree-of-Freedom Robust $H_\infty$ Repetitive,'' In \emph{Proceedings of The 2014 International Conference on Electrical Engineering/Electronics,Computer, Telecommunications and Information Technology (ECIT-CON 2014)}, Nakhon Ratchasima, Thailand, May 14--17, 2014,
    \item W. Sriart, \textbf{S. Boonto}, S. Naetiladdanon, W. Lenwari, ``Two-Degree-of-Freedom Robust $\mathcal{H}_\infty$ Repetitive Control for Grid-Connected Inverter,'' In \emph{Proceedings of the 11th IEEE International Conference on Control and Automation (ICCA 2014)}, Taichung, Taiwan, June 18-20, 2014, pp. 791--796
    \item N. Patcharaprakiti, K. Kirtikara, A. Sanswang, \textbf{S. Boonto}, ``Stability Analysis of a Photovoltaic Grid Connected Inverter Model Based on System Identification,'' In \emph{Proceedings of the 2012 IEEE Region 10 Conference}, Cebu, Philippines, November 19--22, 2012, pp. 1--4.
    \item {A. Kominek, S. Remolina, \textbf{S. Boonto}, H.  Werner, M. Garwon, and M. Schultalbers, ``Low-Complexity LPV Input-Output Identification and Control of a Turbocharged Combustion Engine,'' In \emph{Proceedings of the 51th IEEE Conference on Decision and Control}, Maui, HI., USA, December 10--13, 2012, pp. 4492--4497.}
    \item {\textbf{S. Boonto} and W. Lenwari, ``Two-Degree-of-Freedom $\mathcal{H}_{\infty}$ Control Design for Harmonic Current Control of Shunt Active Filters,'' In \emph{Proceedings of the 15th IEEE International Conference on Harmonics and Quality of Power (ICHQP 2012)}, Hong Kong, June 2012, pp. 887--891}
    \item Q. Liu, G. Kaiser, \textbf{S. Boonto}, H. Werner, F. Holzmann, B. Chertien, M. Korte, ``Two-Degree-of-Freedom LPV Control for a through-the-Road Hybrid Electric Vehicle via Torque	Vectoring,'' In \emph{Proceedings of the 50th IEEE Conference on Decision and Control and European Control Conference -- CDC--ECC 2011}, Orlando, FL, USA, December 12--15, 2011, pp. 1274--1279.
	\item I. Wior, \textbf{S. Boonto}, H. Abbas, H. Werner, ``Modeling and Control of an Experimental pH Neutralization Plant using Neural Networks based Approximate Predictive Control,'' In \emph{Proceedings of  the 1st Virtual Control Conference}, Denmark, 22 Sep, 2010, (online).
    \item \textbf{S. Boonto}, H. Werner, ``Closed-Loop Identification of LPV Models Using Cubic Splines with Application to an Arm-Driven Inverted Pendulum,'' In \emph{Proceedings of the 2010 American Control Conference -- ACC2010}, Baltimore, Maryland, USA, June 30 - July 2, 2010, pp. 3100--3105
    \item \textbf{S. Boonto}, H. Werner, ``Closed-Loop System Identification of LPV Input-Output Models -- Application to an Arm-Driven Inverted Pendulum,'' In \emph{Proceedings of the 47th IEEE Conference on Decision and Control}, Cancun, Mexico , December 9--11, 2008, pp. 2606--2611
    \item J. Witt, \textbf{S. Boonto}, H. Werner, ``Approximate Model Predictive Control of a 3-DOF Helicopter,'' In \emph{Proceedings of the 46th IEEE Conference on Decision and Control}, New Orleans, Louisiana USA, December 12--14, 2007,  pp. 4501--4506
    \item O. Supatti, \textbf{S. Boonto}, C. Prapanavarat, V. Moneyagul,
        ``Design of an $\mathcal{H}_\infty$ Robust Controlled for Multimodule
        Parallel DC-DC Buck Converters with Average Current Mode Control,''
        In \emph{Proceedings of IEEE International Conference on Industrial Technology}
        , Bangkok, Thailand,  December 11--14, 2002, pp. 992-997.

    \item O. Supatti, \textbf{S. Boonto}, C. Prapanavarat, V. Moneyagul,
        ``$\mathcal{H}_\infty$ Controller
        Design for Parallel DC-DC Buck Converters,'' In \emph{Proceedings of 17th Korea
        Automatic Control Conf.}, Jeonbuk, Korea, October 16--19, 2002, pp. 1159�-1163.
\end{enumerate}
\textbf{National Conference}
\begin{enumerate}
    \item N. Dernlugkam, \textbf{S. Boonto}, P. Siriprala, ``Identification and Control of a Half-scale Platform of Multi-Launcher Rocket System,'' In \emph{Proceedings of the 36th Electrical Engineering Conference (EECON36)}, Kanchanaburi, Thailand, December 11-13, 2013. (in Thai)
	\item C. Techawatcharapaikul, \textbf{S. Boonto},``PI/PID Design and Tuning
        via LMI with Time Domain Constraint,'' In \emph{Proceedings of the 41st
        Kasetsart University Annual Conference}, Bangkok, Thailand,
        February 2003. (in Thai)
    \item S. Teratanajaru, \textbf{S. Boonto}, A. Chaisawadi,``On-line
        area-based computation method for first-order plus dead-time model
        system identification from step response,'' In \emph{Proceedings of the
        17th Conference on Mechanical Engineering Network of Thailand},
        Prajeanburi, Thailand, October 2003.
    \item P. Sritanauthaikarn,\textbf{ S. Boonto}, A. Chaisawadi,``Linear
        Matrix Inequalities Based Controller Design for Crane System,''
        In \emph{Proceedings of the 17th Conference on Mechanical Engineering
        Network of Thailand}, Prajeanburi, Thailand, October 2003.
        (in Thai)
\end{enumerate}
%\section{Academic Experience}
%\href{http://www.osu.edu}{\textbf{The Ohio State University}},
%Columbus, Ohio USA
%\begin{outerlist}
%\item[] \textit{Graduate Student}%
%        \hfill \textbf{June 2004 to present}
%\begin{innerlist}
%\item \href{http://www.gradsch.osu.edu/Content.aspx?Content=44&itemid=2}
%           {Dean's Distinguished University Fellow}
%      (June 2004 to present)
%        \begin{innerlist}
%        \item[] Includes current M.S.~research and course work.
%        \end{innerlist}
%\item \href{http://www.nsfgk12.org/}
%           {National Science Foundation GK-12 Fellow}
%      (September 2006 to October 2007)
%        \begin{innerlist}
%        \item[] Developed, implemented, and evaluated daily fourth grade
%                science lessons for a local inner-city public school
%                class.
%        \end{innerlist}
%\end{innerlist}
%
%\item[] \textit{Instructor}%
%        \hfill \textbf{March 2002 to June 2004}
%\begin{innerlist}
%\item Member of \href{http://feh.eng.ohio-state.edu/}
%                     {Fundamentals of Engineering for Honors}
%      instructional team.
%\item Special graduate teaching appointment as undergraduate.
%\item Lectured weekly laboratory on engineering fundamentals (ENG H191,
%        H192, and H193).
%\item Trained in-class undergraduate teaching assistants in laboratory
%        procedure.
%\item Graded weekly lab reports and provided laboratory exams.
%\end{innerlist}
%
%\item[] \textit{Teaching Assistant}%
%        \hfill \textbf{September 2000 to March 2002}
%\begin{innerlist}
%\item Assisted \href{http://feh.eng.ohio-state.edu/}
%                    {Fundamentals of Engineering for Honors}
%      instructional team.
%\item Provided in-class support to first-year engineering students (ENG
%        H191, H192, and H193).
%\item Graded daily assignments on programming and drafting.
%\end{innerlist}
%
%\item[] \textit{Undergraduate Researcher}%
%        \hfill \textbf{September 2000 to March 2002}
%\begin{innerlist}
%\item Participated in the
%        \href{http://www.cse.ohio-state.edu/europa/}{Europa
%        Undergraduate Research Forum}, a part of the
%        \href{http://www.cse.ohio-state.edu/rsrg/}{Reusable Software
%        Research Group}.
%\item Worked to improve undergraduate education of component based
%        software engineering topics.
%\item Researched needed changes to RESOLVE/C++ implementation for
%        ANSI/C++ compliance.
%\end{innerlist}
%
%\item[] \textit{Grader}%
%        \hfill \textbf{September 2001 to December 2001}
%\begin{innerlist}
%\item Graded daily electromagnetics assignments (ECE 311).
%\end{innerlist}
%
%\item[] \textit{Undergraduate Student}%
%        \hfill \textbf{September 1999 to June 2004}
%\end{outerlist}
%
%\section{Publications}
%%
%Pavlic, T.P., and K.M.~Passino. Submitted. Foraging Theory for Mobile
%Agent Speed Choice. \href{http://www.elsevier.com/locate/engappai}
%                         {Engineering Applications of Artificial
%                         Intelligence}.
%
%\section{Books in Preparation}
%%
%Pavlic, T.P., B.W.~Andrews, K.M.~Passino, and T.A.~Waite. Foraging
%Theory for Engineering.
%
%\section{Conference Publications}
%%
%Freuler, R.J., M.J.~Hoffmann, T.P.~Pavlic, J.M.~Beams, J.P.~Radigan,
%P.K.~Dutta, J.T.~Demel, and E.D.~Justen. 2003. Experiences with a
%Comprehensive Freshman Hands-On Course -- Designing, Building, and
%Testing Small Autonomous Robots. Proceedings of the 2003
%\href{http://www.asee.org/}{American Society for Engineering Education}
%Annual Conference \& Exposition.
%
%\section{Professional Experience}
%%
%\href{http://www.ni.com/}{\textbf{National Instruments}},
%Austin, Texas USA
%\begin{outerlist}
%
%\item[] \textit{Hardware R\&D Intern for Multifunction DAQ}%
%        \hfill \textbf{June 2003 to September 2003}
%\begin{innerlist}
%\item Designed final verification testing fixture for use with STC2 MIO
%        products.
%\item Designed and executed study of the effect of varying burn-in time
%        on long-term drift of common industry voltage references.
%\end{innerlist}
%
%\item[] \textit{Hardware R\&D Intern for Multifunction DAQ}%
%        \hfill \textbf{June 2002 to September 2002}
%\begin{innerlist}
%\item Designed and performed validation tests on new 16-bit 800 kHz
%        NI-6120 SMIO DAQ board.
%
%\item Designed high quality filter/amplifier source for use with NI-5411
%        arbitrary function generator.
%\end{innerlist}
%
%\end{outerlist}
%
%\blankline
%
%\textbf{\href{http://www.ibm.com/}{IBM} Network Storage},
%Research Triangle Park, North Carolina USA
%\begin{outerlist}
%
%\item[] \textit{Core Systems Software Developer for FlexNAS}%
%        \hfill \textbf{June 2001 to September 2001}
%\begin{innerlist}
%\item Designed and implemented high-availability, redundant internode
%        communications subsystem.
%\item Participated in software development of various vital box
%        services.
%\end{innerlist}
%
%\end{outerlist}
%
%\blankline
%
%\href{http://www.calltech.com/}{\textbf{CallTech Communications}},
%Columbus, Ohio USA
%\begin{outerlist}
%
%\item[] \textit{Information Technology Systems Engineer}%
%        \hfill \textbf{June 1997 to May 2001}
%\begin{innerlist}
%\item Responsible for the acquisition, setup, maintenance, and
%        administration of all Internet hardware and software supporting
%        \href{http://www.netwalk.com/}{NetWalk} Internet service
%        and web presence provider.
%\item Designed and implemented state of the art open source
%        high-availability load balancing system supporting thousands of
%        virtual servers.
%\item Developed software call center support software for clients such
%        as CompuServe, AOL, and Priceline.
%\end{innerlist}
%
%\end{outerlist}
%
%\blankline
%
%MegaLinx Communications, Dublin, Ohio USA
%\begin{outerlist}
%
%\item[] \textit{Web Developer and Support Representative}%
%        \hfill \textbf{June 1995 to May 1997}
%\begin{innerlist}
%\item Produced web content for commercial clients.
%\item Assisted in administration of UltraSPARC, x86, 68020, 68030, and
%        PowerPC systems running Sun Solaris, Linux, Microsoft DOS,
%        Microsoft Windows NT, and Apple Macintosh operating systems.
%\item Developed multi-platform open source file sharing solution.
%\item Provided technical support for Internet and web presence
%        customers.
%\end{innerlist}
%
%\end{outerlist}
%
%\section{Service}
%%
%Director of Computers,
%\href{http://ec.osu.edu/}{Engineers' Council},
%\href{http://www.osu.edu/}{The Ohio State University}, 2002
%
%\blankline
%
%\href{http://www.osufirst.org/}{OSU FIRST Robotics Team},
%\href{http://www.osu.edu}{The Ohio State University}, 2000--2004
%\begin{innerlist}
%\item Introduced middle school and high school students to science and
%        technology by participating with them in national robotics
%        competitions.
%\item Led 2002 team to regional silver medal
%        \href{http://www.firstwiki.org/Engineering_Inspiration_Award}
%             {\emph{Engineering Inspiration Award}}.
%\item \emph{Lead Team Mentor}, 2002--2004
%\item \emph{Component Design Team Lead Mentor}, 2001--2002
%\end{innerlist}
%
%\blankline
%
%\href{http://www.linuxvirtualserver.org/}
%     {Linux Virtual Server Project}, 1999--2000
%\begin{innerlist}
%\item Early member of the team that formed the open source project that
%        is now an important load balancing solution for the Linux
%        software platform.
%\end{innerlist}
%
%\blankline
%
%\href{http://www.gcfn.org/}
%     {Greater Columbus Free-Net}, 1995--1997
%\begin{innerlist}
%\item Provided technical support services.
%\end{innerlist}
%
%\blankline
%
%CompuTeen Bulletin Board System, 1993--1995
%\begin{innerlist}
%\item Administrated dial-up bulletin board system.
%\item Founded and administrated TeenLiNK, an international electronic
%        mail network that spread through the United States, Canada, and
%        Australia and delivered mail over a series of electronic dial-up
%        drop offs.
%\end{innerlist}
%
%
%\section{Technical Skills}
%%
%Extensive hardware and software experience in networking and
%        information technology
%
%\blankline
%
%\href{http://www.mathworks.com/products/matlab/}{\textsc{Matlab}}
%        experience: linear algebra, Fourier transforms,
%        nonlinear numerical methods, polynomials, statistics,
%        visualization
%
%\blankline
%
%\href{http://www.mathworks.com/products/matlab/}{\textsc{Matlab}}
%        toolboxes: communications, control system, filter
%        design, genetic algorithm and direct search, signal processing,
%        system identification
%
%\blankline
%
%Instrumentation and Control:
%        \href{http://www.dspaceinc.com/}{dSPACE} hardware and software,
%        \href{http://www.mathworks.com/products/simulink/}{Simulink},
%        \href{http://www.ni.com/}{LabVIEW} and other
%        \href{http://www.ni.com}{National Instruments}
%        control and data acquisition hardware and software
%
%\blankline
%
%Programming: C, C++, Pascal, Perl, PHP, Lisp, UNIX shell scripting, SQL,
%        RCS, CVS, SVN, and others
%
%\blankline
%
%Applications: \TeX{}, \LaTeX{}, B\textsc{ib}\TeX{}, Microsoft Office,
%        and other common productivity packages for Windows, OS X, and
%        Linux platforms
%
%\blankline
%
%Operating Systems: Microsoft Windows XP/2000, Apple OS X, Linux, BSD,
%        IRIX, AIX, Solaris, and other UNIX variants
%
%\section{Mathematical Expertise}
%%
%Linear and Nonlinear Systems Theory
%
%\blankline
%
%Probability, Random Variables, and Stochastic Processes
%
%\blankline
%
%Dynamic Optimization
%
%\blankline
%
%Game Theory

\end{document}

%%%%%%%%%%%%%%%%%%%%%%%%%% End CV Document %%%%%%%%%%%%%%%%%%%%%%%%%%%%%
